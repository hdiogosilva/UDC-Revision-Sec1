%----------------------------------------------------------------------------------------
%	Conclus�es
%----------------------------------------------------------------------------------------

\chapter*{Reflex�es finais}
  \addcontentsline{toc}{chapter}{Reflex�es finais} \markboth{CONSLUSAO}{}

A proposta de revis�o neste trabalho apresentada segue a linha, n�o s� da revis�o em curso para a �rea da Filosofia, mas tamb�m a linha que outras classes da CDU t�m aplicado, no que diz respeito � implementa��o de classifica��es multifacetadas. Por tudo isto � ent�o de esperar que muitos dos pontos aqui discutidos seja realmente introduzidos na classifica��o em si. � ainda tamb�m necess�rio real�ar que com a revis�o de cada vez mais classes surge a necessidade do processo de sele��o de facetas para cada categoria seja alvo de estudo, de maneira a que este seja padronizado e exista o menor n�mero de discrep�ncias poss�veis entre as tabelas auxiliares de cada classe, surgindo at� a possibilidade da exist�ncia de mais tabelas auxiliares com facetas adicionais comuns a cada classe. \\

Fora do dom�nio deste trabalho esteve ainda o realojamento da classe de Psicologia, sendo que esta deve ser movida para perto das ci�ncias sociais, estando o seu posicionamento concreto ainda por definir. Quanto � classe referente ao paranormal, este seria um caso mais complexo visto que a classe onde melhor se enquadraria seria a classe 2 e esta sofreu a sua revis�o � relativamente pouco tempo.
