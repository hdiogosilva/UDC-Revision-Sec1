%----------------------------------------------------------------------------------------
%	Abstract / Resumo
%!TEX root = ../templ.tex
%----------------------------------------------------------------------------------------

\begin{abstract}
	A classe 1 da Classifica��o Decimal Universal n�o v� uma atualiza��o ao seu conte�do � mais de quarenta e cinco anos, encontrando-se assim desatualizada com a atual literatura sobre este campo. Utilizando as novas tend�ncias de moderniza��o da CDU apresentamos neste trabalho uma proposta para a altera��o desta classe, � semelhan�a da revis�o conduzida � classe \emph{2 Religi�o. Teologia}.\\
	Os principais aspetos desta revis�o passam pela cria��o de uma classifica��o multifacetada para esta classe; a remoca��o dos campos \emph{159.9  Psicologia} e \emph{133  Paranormal. O oculto. Fen�menos psi}, sendo que no caso da psicologia esta deve ser movida para uma classe mais pr�xima das ci�ncias sociais, e no caso do campo 133, considerada a sua mudan�a para a classe 2; adi��o de novos conceitos � classe, de maneira a melhor representar a bibliografia atual desta �rea; alterar a classe ao n�vel fundamental, de maneira a permitir a classifica��o bibliogr�fica de qualquer filosofia, independentemente da sua origem.
	\\
	\\
	\\
	\\
	\\
	\\
	\textbf{Palavras-chave:} Classifica��o Decimal Universal, Filosofia, Classifica��o
\end{abstract}
