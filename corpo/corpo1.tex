%----------------------------------------------------------------------------------------
%	Corpo do trabalho
%----------------------------------------------------------------------------------------

\chapter{Capitulo1}

\tikzstyle{every node}=[anchor=west]
\tikzstyle{selected}=[draw=red,fill=red!30]
\tikzstyle{optional}=[dashed,fill=gray!50]

\begin{tikzpicture}[%
scale=.7,
grow via three points={one child at (0.5,-0.65) and
	two children at (0.5,-0.65) and (0.5,-1.3)},
edge from parent path={(\tikzparentnode.south) |- (\tikzchildnode.west)}]

\node {1 FILOSOFIA. PSICOLOGIA.}
child { node {101 Natureza e �mbito da filosofia}}		
child { node {11  Metaf�sica}
	child { node {111 Metaf�sica geral. Ontologia}}
	child { node {113/119 Cosmologia. Filosofia da natureza}}
}
child [missing] {}				
child [missing] {}
child { node {122/129 Metaf�sica especial}}
child { node {13 Filosofia da mente e do esp�rito. Metaf�sica da vida espiritual}
	child { node {130.1 Conceitos e leis gerais}}
	child { node {130.2 Filosofia da cultura. Sistemas culturais. Teoria dos complexos culturais}}
	child { node {130.3  Metaf�sica da vida espiritual}}
	child { node {133  Paranormal. O oculto. Fen�menos psi}}
}
child [missing] {}				
child [missing] {}				
child [missing] {}
child [missing] {}					
child { node {14 Sistemas e pontos de vista filos�ficos}
	child { node {140 Atitudes filos�ficas poss�veis. Tipologia de sistemas filos�ficos}}
	child { node {141 Tipos de pontos de vista filos�ficos}}
}
child [missing] {}
child [missing] {}
child { node {159.9 Psicologia}
	child { node {159.91 Psicofisiologia (psicologia fisiol�gica). Fisiologia mental}}
	child { node {159.92 Desenvolvimento e capacidade mental. Psicologia comparada}}
	child { node {159.93 Sensa��o. Percep��o sensorial}}
	child { node {159.94 Fun��es executivas}
		child { node {159.942 Emo��es. Afectos. Sensibilidade. Sentimentos}}
		child { node {159.943 Cona��o e movimento}}
		child { node {159.944 Trabalho e fadiga. Efici�ncia}}
		child { node {159.946 Fun��es motoras especiais}}
		child { node {159.947 Voli��o. Vontade}}
	}
	child [missing] {}
	child [missing] {}
	child [missing] {}
	child [missing] {}
	child [missing] {}
	child { node {159.95 Processos mentais superiores}}
	child { node {159.96 Estados e processos mentais especiais}}
	child { node {159.97 Psicopatologia}}
	child { node {159.98 Psicologia aplicada (psicotecnologia) em geral}}
}
child [missing] {}
child [missing] {}
child [missing] {}
child [missing] {}
child [missing] {}
child [missing] {}
child [missing] {}
child [missing] {}
child [missing] {}
child [missing] {}
child [missing] {}
child [missing] {}
child [missing] {}
child { node {16 L�gica. Epistemologia. Teoria do conhecimento. Metodologia da l�gica}
	child { node {161/162 Fundamentos da l�gica}}
	child { node {164 Log�stica. L�gica simb�lica. L�gica matem�tica. C�lculo l�gico}}
	child { node {165 Teoria do conhecimento. Epistemologia}
		child { node {165.6/.8 Pontos de vista e doutrinas epistemol�gicas}}
	}
	child [missing] {}
	child { node {167/168 Metodologia l�gica}}
}
child [missing] {}
child [missing] {}
child [missing] {}
child [missing] {}
child [missing] {}	
child { node {17 Filosofia moral. �tica. Filosofia pr�tica}
	child { node {171 �tica individual. Deveres do indiv�duo para consigo mesmo}}
	child { node {172 �tica social. Deveres para com os outros}}
	child { node {173 �tica familiar}}
	child { node {176 �tica sexual. Moralidade sexual}}
	child { node {177 �tica e sociedade}}
};
\end{tikzpicture}