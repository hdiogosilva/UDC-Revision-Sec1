%----------------------------------------------------------------------------------------
%	Corpo do trabalho
%----------------------------------------------------------------------------------------

\chapter{Capitulo2}

\begin{description}
  \item[1]{FILOSOFIA}
  \begin{description}
    \item[1-1/-6]{Subdivis�es auxiliares especiais para a filosofia}
      \begin{description}
        \item[1-1]{Fontes. Materiais}
        \item[1-2]{Pessoas na filosofia}
        \item[1-3]{Aplica��es. Filosofia aplicada}
        \item[1-4]{Pr�tica. M�todos. Argumenta��o}
        \item[1-5]{Pontos de vista, doutrinas, abordagens, teorias, atitudes, ismos}
        \item[1-6]{Sistemas. Escolas. Tradi��es. Periodos. Hist�ria}
      \end{description}
    \item[11/18]{Ramos. Disciplinas}
      \begin{description}
        \item[11]{Metaf�sica geral}
        \item[12]{Metaf�sica especial}
        \item[13]{Filosofia da mente}
        \item[14]{Filosofia da linguagem}
        \item[15]{Est�tica. Filosofia est�tica}
        \item[16]{L�gica. Epistemologia. Teoria do conhecimento. Metodologia da l�gica}
        \item[17]{Filosofia moral. �tica. Filosofia pr�tica}
        \item[18]{Filosofia politica. Filosofia da lei}
      \end{description}
    \end{description}
\end{description}

\begin{table}
  \centering
  \begin{tabular}{ r l l }
    \rowcolor[gray]{0.9}
    11/18 & Ramos. Disciplinas & \textbf{\emph{[Things]}} \\
    \ldots & \ldots & \ldots \\
    \rowcolor[gray]{0.9}
    {0.9}1-1/-6 & Subdivis�es auxiliares especiais para a filosofia & \\
    1-1 & Fontes. Materiais & \textbf{\emph{[Materials]}} \\
    1-2 & Pessoas na filosofia & \textbf{\emph{[Agents]}} \\
    1-3 & Aplica��es. Filosofia aplicada & \textbf{\emph{[Patients]}} \\
    1-4 & Pr�tica. M�todos. Argumenta��o & \textbf{\emph{[Operations]}} \\
    1-5 & Pontos de vista, doutrinas, abordagens, teorias, atitudes, ismos & \textbf{\emph{[Properties]}} \\
    1-6 & Sistemas. Escolas. Tradi��es. Periodos. Hist�ria & \textbf{\emph{[Kinds]}} \\
  \end{tabular}
  \caption{Correspond�ncia entre as facetas identificadas e as categorias de facetas defenidas pela classifica��o bibliogr�fica de Bliss}
  \label{tabela:1}
\end{table}

\begin{table}
  \centering
  \begin{tabular}{ c r l p{5cm} }
    & \textbf{Categoria} & \textbf{Zona/auxiliares correspondentes} \\[1ex]
    \parbox[t]{2mm}{\multirow{11}{*}{\rotatebox[origin=c]{90}{Como apresentadas por \citet{vickery1960}}}} & \emph{[Thing]} & 11/18 \\[0.5ex]
    & \emph{[Material]} & -1 \\[0.5ex]
    & \emph{[Agent]} & -2 \\[0.5ex]
    & \emph{[Patient]} & -3 \\[0.5ex]
    & \emph{[Operation]} & -4 \\[0.5ex]
    & \emph{[Property]} & -5 \\[0.5ex]
    & \emph{[Kind]} & -6 \\[0.5ex]
    & \emph{[Time]} & ``1/9'' \\[0.5ex]
    & \emph{[Space]} & (1/9) \\[0.5ex]
    & \emph{[Process]} & \emph{N�o Atribuido} \\[0.5ex]
    & \emph{[Part]} & \emph{N�o Atribuido} \\[0.5ex]
  \end{tabular}
  \caption{Categorias de facetas defenidas pela classifica��o bibliogr�fica de Bliss utilizadas}
  \label{tabela:2}
\end{table}
